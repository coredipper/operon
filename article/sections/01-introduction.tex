\section{Introduction}

Artificial Intelligence is shifting from \emph{Generative AI} (single-shot text generation) to \emph{Agentic AI}
(systems that execute multi-step workflows to achieve goals). While individual LLM capabilities scale
predictably, engineering \emph{systems of agents} remains fragile. Developers face non-deterministic
outputs, runaway loops, adversarial instructions, and the challenge of maintaining global coherence
under partial observability and stochastic behavior.

We argue these challenges are not unique to software. They are fundamental constraints of distributed
information processing. A close analogue to a multi-agent architecture is a \emph{Gene Regulatory Network}
(GRN): thousands of genes read local signals and express proteins that regulate other genes, producing
robust behavior under noise, resource constraints, and adversarial environments.

\subsection{The Biological Heuristic}

Systems biology identifies recurring \emph{network motifs} that implement control functions (filtering,
memory, consensus, and defense). We focus on four motifs that map naturally to agentic engineering:

\begin{itemize}[leftmargin=*]
  \item \textbf{Coherent Feed-Forward Loop (CFFL):} persistence detection and gating, analogous to
  two-key execution guardrails.
  \item \textbf{Quorum sensing:} distributed consensus via thresholded aggregation, analogous to
  ensemble voting with confidence thresholds.
  \item \textbf{Chaperone-assisted folding:} structural validation and repair loops, analogous to schema
  validation and retry mechanisms.
  \item \textbf{Immune-like self-defense:} mechanisms distinguishing trusted signals from untrusted
  inputs, analogous to prompt-injection containment and information-flow policies.
\end{itemize}

\subsection{From Metaphor to Discipline}

To make the analogy actionable, we use Applied Category Theory:
polynomial functors in $\Poly$ represent \emph{typed open interfaces}, and wiring diagrams represent
\emph{compositional structure}. We emphasize an important precision:

\begin{quote}
We do \emph{not} claim GRNs and agentic systems are identical in physics or implementation.
We claim they admit a shared \emph{interface-level semantics} in $\Poly$ that supports compositional
reasoning about wiring, validation, gating, and resource control.
\end{quote}

\subsection{Contributions}

\begin{enumerate}[leftmargin=*]
  \item \textbf{A Formal Dictionary:} a rigorous correspondence between biological components
  (genes, promoters, plasmids) and software components (agents, schemas, tools).
  \item \textbf{A Safer Agentic Operad $\WAgent$:} a syntax for agent wiring with types, integrity labels,
  and effect tracking, supporting compile-time rejection of ill-formed compositions.
  \item \textbf{Correct Topological Safety Claims:} topology enforces \emph{interlocks}, not independence;
  we provide the correct probabilistic statement for CFFL safety and the corresponding design
  requirements for independence/diversity.
  \item \textbf{A Pathology Lens:} agentic failures are analyzed as dysregulated network dynamics,
  informing termination, validation, and security treatments.
\end{enumerate}

