\section{A Formal Syntax for Agent Composition: The Operad \texorpdfstring{$\WAgent$}{WAgent}}

We now define a wiring discipline for agentic systems that supports:
(1) static rejection of ill-typed wiring,
(2) explicit validation boundaries, and
(3) information-flow and capability reasoning.

\subsection{Data Types, Integrity Labels, and Capabilities}

\begin{definition}[Base Data Types]
Let $\mathcal{T}$ be a set of base data types, e.g.
\[
  \mathcal{T} = \{\mathsf{Text},\, \mathsf{JSON}(\Sigma),\, \mathsf{Image},\, \mathsf{ToolCall}(\kappa),\, \mathsf{Error},\, \mathsf{Stop},\, \Approval\}.
\]
\end{definition}

\begin{definition}[Integrity Labels]
Let $\mathcal{L}=\{\mathsf{U},\mathsf{V},\mathsf{T}\}$ denote integrity levels:
\[
  \mathsf{U}=\text{untrusted},\quad
  \mathsf{V}=\text{validated},\quad
  \mathsf{T}=\text{trusted}.
\]
We order them $\mathsf{U} \le \mathsf{V} \le \mathsf{T}$ and enforce \emph{integrity-preserving flow}:
a wire may connect output label $\ell_{\mathrm{out}}$ to input label $\ell_{\mathrm{in}}$ only if
\begin{equation}
  \ell_{\mathrm{out}} \ge \ell_{\mathrm{in}}.
  \label{eq:integrity-flow}
\end{equation}
Thus untrusted data cannot directly feed a trusted-required port.
\end{definition}

\begin{definition}[Capabilities / Effects]
Let $\mathcal{C}$ be a set of capabilities (effects), e.g.
\[
  \mathcal{C}=\{\mathsf{ReadFS},\mathsf{WriteFS},\mathsf{Net},\mathsf{ExecCode},\mathsf{Money},\mathsf{EmailSend}\}.
\]
Each box (agent/module) carries an effect set $\epsilon \subseteq \mathcal{C}$, and composition
aggregates effects by union unless restricted by a policy.
\end{definition}

\subsection{Ports and Well-Formed Wiring}

\begin{definition}[Port Type]
A port carries a \emph{decorated type} $(\tau,\ell)$ where $\tau\in\mathcal{T}$ and $\ell\in\mathcal{L}$.
(Effects are box-level annotations, but may be required by specific ports such as execution.)
\end{definition}

\begin{definition}[Well-Typed Wire]
A wire from port $(\tau_1,\ell_1)$ to port $(\tau_2,\ell_2)$ is well-typed when:
\[
  \tau_1 = \tau_2
  \quad\text{and}\quad
  \ell_1 \ge \ell_2.
\]
\end{definition}

\begin{remark}
This is the core distinction between \emph{structural correctness} and semantic correctness.
$\WAgent$ can prevent structural integration errors (wrong types, wrong trust boundaries) but cannot,
by itself, guarantee the semantic truth of a text produced by an LLM.
\end{remark}

\subsection{Primitive Composition Operations}

The operad provides three key wiring primitives.

\subsubsection{Parallel Composition $(\otimes)$}

Two agents $A$ and $B$ execute in parallel. Parallel composition is safe when they do not race on a
shared mutable resource, or when shared resources are mediated by explicit resource boxes (locks,
transactions, CRDTs, etc.).

\subsubsection{Serial Composition $(\circ)$}

The output of $A$ feeds the input of $B$. In $\WAgent$, ill-typed connections are rejected at design time.
This moves \emph{structural mismatch failures} (e.g., feeding text into a JSON port) from runtime to
architecture time. It does \emph{not} eliminate semantic hallucinations.

\subsubsection{Feedback / Trace $\mathrm{Tr}$}

Feedback loops represent agency and self-reference, but require termination governance.
Rather than appealing to an undefined ``contractive map'' property, we introduce an explicit progress
measure and budget.

\begin{definition}[Guarded Trace (Fuel/Budget)]
A guarded trace $\mathrm{Tr}_B(A)$ wires feedback through a budget box $B$ that enforces:
(1) a maximum number of iterations and/or tokens, and
(2) a decreasing ranking/progress measure unless external evidence increases it.
\end{definition}

