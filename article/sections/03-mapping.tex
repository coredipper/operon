\section{The Bridge: GRNs and Agents as Open Interfaces in \texorpdfstring{$\Poly$}{Poly}}

To relate GRNs and agentic systems, we map both to a common mathematical representation:
polynomial functors for interfaces and coalgebras for stateful behavior.

\subsection{Preliminaries: Polynomial Functors}

\begin{definition}[Polynomial Functor]
A (finitary) polynomial functor $P$ is an object of $\Poly$ of the form
\begin{equation}
  P(y) \;=\; \sum_{o \in O} y^{I(o)} .
  \label{eq:poly}
\end{equation}
Here $O$ is a set of \emph{positions} (outputs). For each $o \in O$, $I(o)$ is the set of
\emph{directions} (inputs) enabled/required after emitting $o$.
\end{definition}

Intuitively, a system exposes an output $o$ and then awaits an input $i \in I(o)$ before proceeding.

\begin{figure}[h]
\centering
\begin{tikzpicture}[>=Latex, node distance=10mm]
  \node[draw, rounded corners, minimum width=28mm, minimum height=10mm] (cap) {$o \in O$};
  \node[draw, circle, below left=10mm and 10mm of cap] (i1) {$i_1$};
  \node[draw, circle, below=10mm of cap] (i2) {$i_2$};
  \node[draw, circle, below right=10mm and 10mm of cap] (i3) {$i_3$};
  \draw[->] (i1) -- (cap);
  \draw[->] (i2) -- (cap);
  \draw[->] (i3) -- (cap);
  \node[below=15mm of i2] {\small A ``corolla'' interface: output then enabled inputs.};
\end{tikzpicture}
\caption{A visual mnemonic for a polynomial interface.}
\end{figure}

\subsection{Coalgebras: Stateful Open Systems}

\begin{definition}[Coalgebra for $P$]
A (deterministic) open dynamical system with interface $P$ is a coalgebra $(S,\varphi)$ where
\begin{equation}
  \varphi: S \to P(S).
  \label{eq:coalgebra}
\end{equation}
Expanding $P(S)=\sum_{o\in O} S^{I(o)}$, the structure map yields:
(1) a \emph{readout} selecting an output position $o$, and
(2) an \emph{update} selecting next state given an input direction.
\end{definition}

This is the level at which both GRNs and agentic systems admit a shared description: state $S$,
interface $P$, and compositional wiring.

\subsection{Two Domain Categories and Two Encodings}

We make the scope of the correspondence explicit.

\begin{definition}[Interface Categories (Sketch)]
Let $\mathbf{GRN}_{\mathrm{int}}$ be a category whose objects are \emph{open regulatory modules}
with typed signal ports, and whose morphisms are admissible couplings (regulatory connections)
that respect those ports. Let $\mathbf{AG}_{\mathrm{int}}$ be a category whose objects are \emph{open agent
modules} with typed I/O ports, and whose morphisms are admissible dataflow connections.
\end{definition}

\begin{proposition}[Encodings into $\Poly$ (Informal)]
There exist interface encodings (functors)
\[
  F: \mathbf{GRN}_{\mathrm{int}} \to \Poly
  \qquad\text{and}\qquad
  G: \mathbf{AG}_{\mathrm{int}} \to \Poly
\]
that map each module to a polynomial interface and each admissible coupling to a morphism in $\Poly$.
\end{proposition}

\begin{remark}
We do not claim $\mathbf{GRN}_{\mathrm{int}} \cong \mathbf{AG}_{\mathrm{int}}$ as categories.
Instead, we claim both admit a faithful \emph{interface-level} representation in $\Poly$ under the
chosen abstraction (typed ports, open composition, and stateful coalgebra semantics).
\end{remark}

\subsection{Genes and Agents as Polynomial Interfaces}

The original presentation used uniform direction sets, which is too coarse for both domains.
We refine the definitions by allowing \emph{position-dependent directions}.

\begin{definition}[Gene Interface Object]
Fix a gene/module $G$. Let $O_G$ be the set of possible expression outcomes (e.g., protein isoforms,
expression regimes, or ``off''), and for each $o\in O_G$ let $I_G(o)$ be the set of regulatory contexts
(binding configurations, signal tuples) that are meaningful after $o$ is selected.
Define:
\begin{equation}
  P_G(y) \;=\; \sum_{o \in O_G} y^{I_G(o)} .
  \label{eq:gene-poly}
\end{equation}
\end{definition}

\begin{definition}[Agent Interface Object]
Fix an agent module $A$. Let $O_A$ be the set of action-positions (e.g., respond to user, call tool
$\kappa$, request clarification, emit approval token, terminate), and for each $o\in O_A$ let $I_A(o)$
be the expected observation type(s) after taking action $o$.
Define:
\begin{equation}
  P_A(y) \;=\; \sum_{o \in O_A} y^{I_A(o)} .
  \label{eq:agent-poly}
\end{equation}
\end{definition}

This refinement matches reality: different actions induce different expected next observations; and
different gene expression regimes correspond to different relevant regulatory inputs.

\subsection{Promoters and Schemas as \emph{Validated} Optics (Not ``Undefined'')}

In biology, a promoter gates what signals are relevant to a gene. In software, schemas and context
windows gate what data is visible and how it is interpreted. The crucial correction is that mismatches
should not be treated as ``mathematically undefined''; they should be modeled as \emph{partial} or
\emph{effectful} maps with explicit error values.

\begin{definition}[Validated Lens (Sketch)]
A validated lens from global state $S$ to local view $V$ is a pair
\[
  \mathrm{get}: S \to \Either(\Err, V),
  \qquad
  \mathrm{put}: S \times V' \to S,
\]
where $\mathrm{get}$ may fail with an error value when the promoter/schema does not match.
\end{definition}

\begin{remark}
This aligns with real agent systems: a schema mismatch typically triggers error handling, retries,
or safe termination, rather than an ``undefined'' computation.
\end{remark}

\subsection{Epigenetics and Memory as State}

Both domains are stateful:
epigenetic markers bias gene responsiveness without changing DNA, while RAG/history bias agent
behavior without changing model weights. In our coalgebra semantics, these correspond to the internal
state $S$.

\subsection{Dictionary}

\begin{table}[h]
\centering
\small
\begin{tabular}{@{}lll@{}}
\toprule
Category Concept & Biological Realization (GRN) & Software Realization (Agentic)\\
\midrule
Polynomial functor $P$ & Module interface & Agent/module interface\\
Position set $O$ & Expression outcomes & Action kinds (tool call, reply, stop, \dots)\\
Directions $I(o)$ & Regulatory contexts & Expected next observation types\\
Validated lens & Promoter gating & Schema/context gating with error handling\\
State $S$ & Epigenetic landscape & Memory/RAG/history\\
Morphisms & Signal transduction wiring & Dataflow wiring\\
\bottomrule
\end{tabular}
\caption{A compositional dictionary at the interface level.}
\end{table}

