\section{Related Work}

This work sits at the intersection of Systems Biology, Applied Category Theory, and Agentic AI.

\subsection{Network Motifs in Systems Biology}

Network motifs as statistically over-represented subgraphs were introduced by Milo et al.
Their functional characterization, including feed-forward persistence detection, is developed further
by Alon.

\subsection{Applied Category Theory (ACT)}

We use the categorical language of open dynamical systems, polynomial functors, and wiring-diagram
operads (e.g., Spivak; Vagner--Spivak--Lerman) to formalize interface composition.

\subsection{Reliability in Agentic AI}

Techniques such as iterative self-critique and reasoning loops improve output quality but typically act
at the prompt level rather than at the topology level. We propose that reliability can be designed as a
property of architecture: gating, validation, resource budgets, and information-flow constraints.

