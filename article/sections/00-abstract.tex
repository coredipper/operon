\begin{abstract}
The transition of Large Language Models (LLMs) from passive generators to autonomous agents
has introduced recurring failures in reliability, security, and state management. Current agentic
architectures are often constructed ad-hoc, leaving systems vulnerable to cascading errors, runaway
feedback loops, and prompt injection attacks.

This paper argues that many of these failure modes are instances of general control problems long
addressed by biological networks. We formalize a \emph{functorial correspondence}---at the level of
\emph{open-system interfaces and composition}---between Gene Regulatory Networks (GRNs) and
Agentic Software Systems using Applied Category Theory. Both are represented as coalgebras of
polynomial functors in $\Poly$, composed via the operad of wiring diagrams.

We introduce $\WAgent$, a typed syntax for agent wiring that (i) prevents structural integration
errors at design time, (ii) enforces explicit validation boundaries through effectful optics and partial
validators, and (iii) supports security reasoning via integrity labels and capability/effect tracking.
Biological motifs---including coherent feed-forward loops (persistence detection), quorum sensing
(consensus), and chaperones (structural validation)---become reusable architectural patterns with
precise interface contracts.

We also revisit a common safety claim: wiring alone does not guarantee statistical independence of
errors. Instead, topology provides \emph{interlocks} (e.g., two-key execution) while independence is an
external design requirement achieved by diversity (models, prompts, tools, and verifiers).

This framework provides a compositional basis for ``epigenetic'' state management (RAG and memory),
for defense against adversarial injection, and for principled termination and resource control in
autonomous software.
\end{abstract}
