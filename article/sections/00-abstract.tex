% ==========================================
% ABSTRACT
% ==========================================
\begin{abstract}
The transition of Large Language Models (LLMs) from passive generators to autonomous agents has introduced significant challenges in reliability, security, and state management. Current agentic architectures are often constructed ad-hoc, prone to ``hallucination cascades,'' infinite loops, and prompt injection attacks. This paper proposes that these failure modes are not unique to software but are instances of universal control problems solved by biological systems over billions of years. We present a formal isomorphism between Gene Regulatory Networks (GRNs) and Agentic Software Systems using \textbf{Applied Category Theory}. We model agents as \textbf{Polynomial Functors} within the category $\Poly$, and their interactions via the \textbf{Operad of Wiring Diagrams}. We derive a rigorous syntax for agent composition by mapping biological mechanisms---including \textit{Quorum Sensing} for consensus, \textit{Chaperone Proteins} for structural validation, and \textit{Endosymbiosis} for neuro-symbolic integration---to software design patterns. This framework provides a mathematical basis for ``Epigenetic'' state management (RAG) and the topological defense against adversarial ``Prion'' attacks.
\end{abstract}
